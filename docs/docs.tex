\documentclass[11pt,fleqn]{book} % Default font size and left-justified equations

\usepackage[top=3cm,bottom=3cm,left=3.2cm,right=3.2cm,headsep=10pt,a4paper]{geometry} % Page margins

\usepackage{xcolor} % Required for specifying colors by name
\definecolor{ocre}{RGB}{243,102,25} % Define the orange color used for highlighting throughout the book

% Font Settings
\usepackage{avant} % Use the Avantgarde font for headings
%\usepackage{times} % Use the Times font for headings
\usepackage{mathptmx} % Use the Adobe Times Roman as the default text font together with math symbols from the Sym­bol, Chancery and Com­puter Modern fonts

\usepackage{microtype} % Slightly tweak font spacing for aesthetics
\usepackage[utf8]{inputenc} % Required for including letters with accents
\usepackage[T1]{fontenc} % Use 8-bit encoding that has 256 glyphs

% Bibliography
% \usepackage[style=alphabetic,sorting=nyt,sortcites=true,autopunct=true,babel=hyphen,hyperref=true,abbreviate=false,backref=true,backend=biber]{biblatex}
% \addbibresource{bibliography.bib} % BibTeX bibliography file
% \defbibheading{bibempty}{}

% Index
\usepackage{calc} % For simpler calculation - used for spacing the index letter headings correctly
\usepackage{makeidx} % Required to make an index
\makeindex % Tells LaTeX to create the files required for indexing

%----------------------------------------------------------------------------------------

\input{structure} % Insert the commands.tex file which contains the majority of the structure behind the template

\newcommand{\itab}[1]{\hspace{0em}\rlap{#1}}
\newcommand{\tab}[1]{\hspace{.2\textwidth}\rlap{#1}}

\begin{document}

\begingroup
\thispagestyle{empty}
  \AddToShipoutPicture*{\put(6,5){\includegraphics[scale=1]{background}}} % Image background
  \centering
  \vspace*{9cm}
  \par\normalfont\fontsize{35}{35}\sffamily\selectfont
  Bad Rear View Mirror Rejection System\par % Book title
  \vspace*{1cm}
  {\Huge Dokumentasi Teknis}\par % Author name
  \vspace*{1cm}
  {\large Laboratorium Rekayasa Fotonika}\par
  {\large Teknik Fisika}\par
  {\large Institut Teknologi Sepuluh November}\par
  {\large 2014}\par
\endgroup

\newpage
~\vfill
\thispagestyle{empty}
% \noindent Copyright \copyright\ 2013 John Smith\\ % Copyright notice
% \noindent \textsc{Published by Publisher}\\ % Publisher
% \noindent \textsc{book-website.com}\\ % URL
\noindent Written by Achmadi @2014 \\
\noindent Licensed under the Creative Commons Attribution-NonCommercial 3.0 Unported License (the ``License''). You may not use this file except in compliance with the License. You may obtain a copy of the License at \url{http://creativecommons.org/licenses/by-nc/3.0}. Unless required by applicable law or agreed to in writing, software distributed under the License is distributed on an \textsc{``as is'' basis, without warranties or conditions of any kind}, either express or implied. See the License for the specific language governing permissions and limitations under the License.\\ % License information
% \noindent \textit{First printing, March 2013} % Printing/edition date

\newpage
\chapterimage{chapter_head_1.pdf} % Table of contents heading image
\pagestyle{empty} % No headers
\tableofcontents % Print the table of contents itself
\cleardoublepage % Forces the first chapter to start on an odd page so it's on the right
\pagestyle{fancy} % Print headers again

\newpage
\chapterimage{chapter_head_2.pdf} % Chapter heading image
\chapter{Development}
\section{Platform}\index{Platform}
\begin{flushleft}
\hspace{10pt} Dalam proyek ini menggunakan platform yang bersifat opensources, kecuali dalam mendesain bagian mekanis.
Keuntungan dalam menggunakan platform yang bersifat opensources antara lain:

\itab{-Gratis dan Legal.}\\
\itab{-Dukungan penuh komunitas.}\\
\itab{-Kemudahan modifikasi software.}\\
\itab{-Kemampuan software opensources untuk development.}\\

\hspace{10pt} Untuk pihak pengembang lain yang berencana ikut mengembangkan dapat mengambil seluruh proyek di repository proyek berbasis Git di alamat:\\
\vspace{5pt}
\url{https://github.com/mekatronik-achmadi/spion\_selector}\\
\vspace{5pt}
\end{flushleft}

\section{Software}\index{Software}
\begin{flushleft}
 \hspace{10pt} Berikut adalah daftar aplikasi yang digunakan dalam proyek ini:\\
 \vspace{5pt}
 \textbf{LinuxMint}\index{LinuxMint}\\
 \hspace{10pt} LinuxMint adalah operating system yang berbasis kernel Linux dan paket manajemen Debian/Ubuntu.
 Disarankan untuk menggunakan sistem Graphical User Interface (GUI) jenis KDE, Cinnamon, maupun MATE.\\
 LinuxMint dapat di download di alamat:\\
 \vspace{5pt}
 \url{http://www.linuxmint.com/download.php}\\
 \vspace{5pt}
 Selain menggunakan LinuxMint, alternatif operating system lain yang disarankan adalah:\\
 \itab{-Ubuntu dengan GUI berjenis Unity. Dapat didownload di \url{http://www.ubuntu.com/download/desktop}}\\
 \itab{-Kubuntu dengan GUI berjenis KDE. Dapat didownload di \url{http://www.kubuntu.org/getkubuntu}}\\
 \itab{-Raspbian yang dijalankan di board RaspberryPi.}\\
 Dapat di didownload di \url{http://www.raspberrypi.org/downloads/}\\
 \vspace{5pt}
 \hspace{10pt} Operating system Raspbian dikhususkan untuk dijalankan di sistem tertanam (Embedded System) berbasis RaspberryPi, bukan untuk komputer/laptop umum.
 Raspbian merupakan operating system berbasis Debian sehingga secara teknis instalasinya akan sama dengan operating system lainnya.\\
 Namun tidak disarankan menggunakan Debian maupun turunan langsungnya (kecuali Raspbian) mengingat versi software yang tersedia kurang update.\\
 \vspace{5pt}
 \textbf{Octave}\index{Octave}\\
 \hspace{10pt} GNU Octave adalah bahasa interpretasi (kebalikan dari kompilasi) level tinggi,utamanya digunakan untuk komputasi numerik. 
 GNU Octave menyediakan kemampuan untuk menemukan solusi numerik baik untuk masalah linear maupun non-linier, termasuk juga untuk experimen numerik. 
 Selain itu juga menyediakan kemampuan grafis yang memadai untuk visualisasi dan manipulasi data. 
 Octave secara umum digunakan melalui interface terminal yang interaktif, tapi dapat pula digunakan untuk program non-interaktif. 
 Bahasa interpretasi Octave sangat mirip dengan Matlab sehingga skrip dari kedua software tersebut dapat saling bertukar.
 Octave memiliki kemampuan setara dengan Matlab,sekalipun belum memiliki fitur seperti Simulink dan GUI.
 
 \hspace{10pt} Octave pertama kali ditulis oleh John W. Eaton ditahun 1988 dalam bahasa C++. 
 Octave dirilis oleh GNU Project, sebuah proyek international pengembangan software gratis dan opensource yang dipimpin oleh R. Stallman. 
 Kini Octave telah menjadi software yang dapat dijalankan di Linux, Windows, Mac,maupun Android. 
 Octave tersedia dalam 19 bahasa berbeda dengan English US sebagai bahasa default. 
 Octave dirilis sebagai software OpenSource dengan lisensi GNU GPL sehingga bersifat legal dan gratis.
 Penjelasan lebih lanjut akan dilanjutkan di bab Octave tersendiri.\\
 
 \vspace{5pt}
 \textbf{OpenCV}\index{OpenCV}\\
 \hspace{10pt} Pustaka OpenCV (Open Computer Vision) merupakan pustaka pemrograman berbasis C/C++/Python yang berisi fungsi-fungsi untuk akuisisi dan pengolahan citra.
  Pustaka OpenCV juga merupakan proyek opensource yang bersifat gratis.
  Saat ini pustaka OpenCV telah diterapkan di banyak website dan aplikasi mobile untuk deteksi wajah dan penjejak warna.
  Dalam proyek ini, OpenCV tidak digunakan untuk mengolah gambar namun sebatas untuk mengambil gambar.
  Bersama dengan Qt (C++), OpenCV ini dibentuk menjadi satu aplikasi untuk berkomunikasi dengan kamera melalui jalur USB dan mengambil gambar.
  Penjelasan lebih lanjut akan dilanjutkan di bersama dalam bab Octave tersendiri.\\
  
 \vspace{5pt}
 \textbf{GCC}\index{GCC}\\
 \hspace{10pt} GCC (GNU Compiler Collection) merupakan koleksi kompiler yang di rilis oleh proyek GNU.
 Kompiler merupakan program yang mengkonversi kode sumber menjadi program siap jalan.
 Saat ini GCC mendukung bahasa C, C++, Java, dan banyak bahasa lain.
 Selain itu GCC juga mendukung banyak arsitektur seperti i386, amd64, armhf, arm-eabi, dan banyak arsitektur lain.
 Dalam proyek ini digunakan dua jenis varian GCC, yaitu g++ dan arm-gcc-none-eabi.
 Varian g++ digunakan untuk mengkompilasi program berbasis OpenCV untuk pengambilan gambar.
 Sedangkan varian arm-gcc-none-eabi digunakan untuk mengkompilasi program yang akan ditanamkan di sirkuit pengendali tangan robot.
 Penjelasan lebih lanjut tentang arm-gcc-none-eabi akan dilanjutkan di bersama dalam bab RTOS tersendiri.\\
 
 \vspace{5pt}
 \textbf{EAGLE}\index{EAGLE}\\
 \hspace{10pt} EAGLE (Easy Applicable Graphical Layout Editor) adalah software untuk mendesain papan circuit.
 Desain yang dibuat di EAGLE berupa skema dan layout.
 Versi yang dapat di instal secara gratis adalah versi Light dengan batasan:\\
 \vspace{5pt}
 \itab{-Luas maximal papan adalah 100mmx80mm.}\\
 \itab{-Hanya dapat menggunakan 2 Lapisan, yaitu Top dan Bottom.}\\
 \itab{-Fitur AutoRoute tidak tersedia.}\\
 \vspace{5pt}
 Namun diluar batasan tadi, fitur EAGLE dapat digunakan secara penuh, terutama ketersedian pustaka komponen yang sangat lengkap.
 Penjelasan lebih lanjut akan dilanjutkan di bab Elektronik tersendiri.\\
 
 \vspace{5pt}
 \textbf{SolidWork}\index{SolidWork}\\
 \hspace{10pt} SolidWork merupakan salah satu software CAD (Computer Aided Design) yang telah populer di perusahan-perusahan besar.
 Software ini digunakan untuk mendesain sistem fisik dan mekanis dengan dukungan pustaka yang sangat lengkap.
 Dalam proyek ini, software ini digunakan untuk mendesain sistem mekanis tangan robot.
 Versi yang digunakan adalah 2013 dan SolidWork tergolong software yang tidak \textit{Backward Compatible}.
 Software ini merupakan satu-satunya software berbayar yang digunakan dalam proyek ini.
 Pembahasan lebih lanjut hanya akan difokuskan pada desain mekanisnya. 
 Namun tentang software tidak dibahas lebih jauh mengingat sifatnya yang tidak gratis dan dijalankan di platform Windows 7.\\
 
\end{flushleft}

\section{License}\index{License}

\newpage
\chapterimage{chapter_head_2.pdf} % Chapter heading image
\chapter{Mekanik}
\section{Gripper}\index{Gripper}

\section{Arm}\index{Arm}

\section{Meja}\index{Meja}

\newpage
\chapterimage{chapter_head_2.pdf} % Chapter heading image
\chapter{Elektronik}
\section{Servo}\index{Servo}

\section{SPS}\index{SPS}

\section{Wiring}\index{Wiring}

\section{PCB}\index{PCB}

\newpage
\chapterimage{chapter_head_2.pdf} % Chapter heading image
\chapter{Controller}
\section{STM32}\index{STM32}

\section{FTDI}\index{FTDI}

\newpage
\chapterimage{chapter_head_2.pdf} % Chapter heading image
\chapter{RTOS}
\section{Threads}\index{Threads}

\section{Commands}\index{Commands}

\newpage
\chapterimage{chapter_head_2.pdf} % Chapter heading image
\chapter{Octave}
\section{Instalasi}\index{Instalasi}

\end{document}


